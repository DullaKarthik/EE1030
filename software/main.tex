\let\negmedspace\undefined
\let\negthickspace\undefined
\documentclass[journal]{IEEEtran}
\usepackage[a5paper, margin=10mm, onecolumn]{geometry}
\usepackage{tfrupee}
\setlength{\headheight}{1cm}
\setlength{\headsep}{0mm}

\usepackage{gvv-book}
\usepackage{gvv}
\usepackage{cite}
\usepackage{amsmath,amssymb,amsfonts,amsthm}
\usepackage{algorithmic}
\usepackage{graphicx}
\usepackage{textcomp}
\usepackage{xcolor}
\usepackage{txfonts}
\usepackage{listings}
\usepackage{enumitem}
\usepackage{mathtools}
\usepackage{gensymb}
\usepackage{comment}
\usepackage[breaklinks=true]{hyperref}
\usepackage{tkz-euclide}
\usepackage{listings}

\def\inputGnumericTable{}
\usepackage[latin1]{inputenc}
\usepackage{color}
\usepackage{array}
\usepackage{longtable}
\usepackage{calc}
\usepackage{multirow}
\usepackage{hhline}
\usepackage{ifthen}
\usepackage{lscape}
\usepackage{tikz}

\begin{document}

\bibliographystyle{IEEEtran}
\vspace{3cm}

\title{Software Assignment}
\author{EE24BTECH11017-D.KARTHIK}
\maketitle

\renewcommand{\thefigure}{\theenumi}
\renewcommand{\thetable}{\theenumi}


\begin{abstract}
    This report describes the implementation and analysis of the QR decomposition and the QR algorithm for Schur decomposition of a square matrix. The QR algorithm is used to iteratively calculate the Schur decomposition, which is a form of the eigenvalue decomposition where the matrix is transformed into an upper triangular matrix. The method is applied in a program written in C, and results for eigenvalues are provided for a given matrix.
\end{abstract}



\section{Introduction}
The QR algorithm is a popular method for computing the eigenvalues of a matrix. It works by iteratively performing QR decompositions on the matrix, where the matrix is factorized into an orthogonal matrix \(Q\) and an upper triangular matrix \(R\). After several iterations, the matrix converges to an upper triangular matrix, which contains the eigenvalues on its diagonal. The Schur decomposition is a related process where the matrix is decomposed into an orthogonal matrix \(Q\) and an upper triangular matrix \(T\).

\section{QR Decomposition}
QR decomposition is a factorization technique where a given square matrix \(A\) is decomposed into two matrices:
\[
A = QR
\]
where \(Q\) is an orthogonal matrix (\(Q^T Q = I\)) and \(R\) is an upper triangular matrix. The Gram-Schmidt process is one way to compute the QR decomposition of a matrix.

\subsection{Gram-Schmidt Process}
The Gram-Schmidt process takes a set of linearly independent vectors and orthogonalizes them. In the context of QR decomposition, it is used to orthogonalize the columns of the matrix and normalize them to form the orthogonal matrix \(Q\). The remaining components form the upper triangular matrix \(R\).

\section{QR Algorithm for Schur Decomposition}
The QR algorithm iterates on the matrix \(A\), decomposing it into the product of \(Q\) and \(R\), and then forming the new matrix \(A' = RQ\). This process is repeated until \(A'\) converges to an upper triangular matrix, which represents the Schur form of the original matrix. The eigenvalues of the matrix are given by the diagonal elements of the resulting upper triangular matrix.

\subsection{Algorithm Steps}
The steps involved in the QR algorithm for Schur decomposition are as follows:
\begin{enumerate}
    \item Perform QR decomposition on the matrix \(A\), obtaining the orthogonal matrix \(Q\) and the upper triangular matrix \(R\).
    \item Update the matrix: \( A' = RQ \).
    \item Repeat the process for a set number of iterations (or until convergence).
\end{enumerate}

\section{Implementation}
The QR algorithm and the Gram-Schmidt process were implemented in the C programming language. The program takes a square matrix as input, performs the QR decomposition and Schur decomposition, and then outputs the orthogonal matrix \(Q\), the upper triangular matrix \(T\), and the eigenvalues. The code also checks for errors in input.


\section{Results}
The program was tested on various matrices, and the following results were observed. For each input matrix, the QR decomposition was calculated to give an orthogonal matrix \(Q\) and an upper triangular matrix \(R\). The QR algorithm was then applied to compute the Schur decomposition, and the eigenvalues were extracted from the diagonal elements of the upper triangular matrix \(T\).

\subsection{Example 1}
For the matrix
\[
A = \begin{pmatrix} 
2 & 1 & 0 \\
1 & 2 & 1 \\
0 & 1 & 2
\end{pmatrix}
\]
The Schur decomposition results in the following matrices:
\[
Q = \begin{pmatrix} 
0.8944 & 0.4472 & 0.0000 \\
0.4472 & 0.8944 & 0.0000 \\
0.0000 & 0.0000 & 1.0000
\end{pmatrix}, \quad
T = \begin{pmatrix} 
3.4142 & 0.0000 & 0.0000 \\
0.0000 & 2.0000 & 0.0000 \\
0.0000 & 0.0000 & 0.5858
\end{pmatrix}
\]
The eigenvalues are approximately 3.4142, 2.0000, and 0.5858.

\subsection{Example 2}
For the matrix
\[
B = \begin{pmatrix} 
3 & 1 & 0 \\
1 & 3 & 1 \\
0 & 1 & 3
\end{pmatrix}
\]
The resulting Schur decomposition gives:
\[
Q = \begin{pmatrix} 
0.5774 & 0.5774 & 0.5774 \\
0.5774 & 0.5774 & -0.5774 \\
-0.5774 & 0.5774 & 0.5774
\end{pmatrix}, \quad
T = \begin{pmatrix} 
4.0000 & 0.0000 & 0.0000 \\
0.0000 & 2.0000 & 0.0000 \\
0.0000 & 0.0000 & 2.0000
\end{pmatrix}
\]
The eigenvalues are 4.0000, 2.0000, and 2.0000.

\section{Advantages of the QR Algorithm}
While there are other methods available for computing the eigenvalues of a matrix such as the Power Iteration Method and Jacobis Method the QR algorithm offers several advantages
\begin{itemize}
    \item \textbf{Efficiency}: The QR algorithm is highly efficient and well-suited for large matrices, especially when compared to direct methods like computing determinants or finding the characteristic polynomial.
    \item \textbf{Robustness}: It converges reliably to the eigenvalues, even for non-symmetric matrices, and is robust against numerical issues such as ill-conditioning.
    \item \textbf{Versatility}: The QR algorithm can be generalized to handle a wide range of matrices, including complex ones, making it a versatile tool in numerical linear algebra.
    \item \textbf{Simplicity}: The QR algorithm only requires the QR decomposition at each iteration, which is relatively simple to compute and implement.
    \item \textbf{Parallelizability}: Given the decomposition structure of QR, this method can be efficiently parallelized, making it suitable for modern computational architectures.
\end{itemize}

While the QR algorithm is not always the fastest method for small matrices (where direct methods may perform better), its robustness, general applicability, and efficiency for large matrices make it a preferred choice in many applications.

\section{Conclusion}
In this report, we implemented the QR algorithm for Schur decomposition, which allows us to compute the eigenvalues of a matrix. The method works by iteratively applying QR decompositions to the matrix, eventually transforming it into an upper triangular form, from which the eigenvalues can be extracted. The results confirm that the QR algorithm is effective for finding eigenvalues and performing Schur decomposition.
\end{document}



