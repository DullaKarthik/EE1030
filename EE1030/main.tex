%iffalse
\let\negmedspace\undefined
\let\negthickspace\undefined
\documentclass[journal,12pt,twocolumn]{IEEEtran}
\usepackage{cite}
\usepackage{amsmath,amssymb,amsfonts,amsthm}
\usepackage{algorithmic}
\usepackage{graphicx}
\usepackage{textcomp}
\usepackage{xcolor}
\usepackage{txfonts}
\usepackage{listings}
\usepackage{enumitem}
\usepackage{mathtools}
\usepackage{gensymb}
\usepackage{comment}
\usepackage[breaklinks=true]{hyperref}
\usepackage{tkz-euclide} 
\usepackage{listings}
\usepackage{gvv}                                        
%\def\inputGnumericTable{}                                 
\usepackage[latin1]{inputenc}                                
\usepackage{color}                                            
\usepackage{array}                                            
\usepackage{longtable}                                       
\usepackage{calc}                                             
\usepackage{multirow}                                         
\usepackage{hhline}                                           
\usepackage{ifthen}                                           
\usepackage{lscape}
\usepackage{tabularx}
\usepackage{array}
\usepackage{float}


\newtheorem{theorem}{Theorem}[section]
\newtheorem{problem}{Problem}
\newtheorem{proposition}{Proposition}[section]
\newtheorem{lemma}{Lemma}[section]
\newtheorem{corollary}[theorem]{Corollary}
\newtheorem{example}{Example}[section]
\newtheorem{definition}[problem]{Definition}
\newcommand{\BEQA}{\begin{eqnarray}}
\newcommand{\EEQA}{\end{eqnarray}}
\newcommand{\define}{\stackrel{\triangle}{=}}
\theoremstyle{remark}
\newtheorem{rem}{Remark}

% Marks the beginning of the document
\begin{document}
\bibliographystyle{IEEEtran}
\vspace{3cm}

\title{Assignment -1}
\author{ee24btech10017-karthik}
\maketitle
\newpage
\bigskip

\renewcommand{\thefigure}{\theenumi}
\renewcommand{\thetable}{\theenumi}
\section{\textbf{Comprehension Based Questions}}
\textbf{ \begin{center} Passage - 2 \end{center} }

Let $A_{1}, G_{1}, H_{1} $ denote the arithmetic, geometric and harmonic means, respectively, of two distinct positive numbers. For $n\geq 2$, Let $A_{n--1}$ and $H_{n--1}$ have arithmetic, geometric and harmonic means as $A_{n},G_{n},H_{n}$ respectively.
  \begin{enumerate}
  \setcounter{enumi}{3}    
  
 \item Which one of the following statements is correct ?
\begin{enumerate}

 
  \item$G_{1}>G_{2}>G_{3}>\dots$ \hfill[2007-4marks]

 \item$G_{1}<G_{2}<G_{3}<\dots$

\item$G_{1}=G_{2}=G_{3}=\dots$

\item$G_{1}<G_{3}<G_{5}<\dots$ and $G_{2}>G_{4}>G_{6}>\dots$
\end{enumerate}

\item Which one of the following statements is correct ?
\begin{enumerate}
    
\item $A_{1}>A_{2}>A_{3}>\dots$ \hfill[2007-4marks]

\item $A_{1}<A_{2}<A_{3}<\dots$

\item $A_{1}>A_{3}>A_{5}>\dots$ and $A_{2}<A_{4}<A_{6}<\dots$

\item $A_{1}<A_{3}<A_{5}<\dots$ and $A_{2}>A_{4}>A_{6}>\dots$
\end{enumerate}

\item Which one of the following statements is correct ?
\begin{enumerate}

 \item $H_{1}>H_{2}>H_{3}>\dots$ \hfill[2007-4marks]

 \item $H_{1}<H_{2}<H_{3}<\dots$

\item $H_{1}>H_{3}>H_{5}>\dots$ and $H_{2}<H_{4}<H_{6}<\dots$

\item $H_{1}<H_{3}<H_{5}<\dots$ and $H_{2}>H_{4}>H_{6}>\dots$
\end{enumerate}
\end{enumerate}
\section{\textbf{Assertion \& Reason Type Questions}}
\begin{enumerate}
    

\item Suppose four positive numbers $a_{1},a_{2},a_{3},a_{4}$ are in G.P. Let $b_{1} = a_{1}, b_{2} = b_{1} + a_{2}, b_{3} = b_{2} + a_{3}$ and $b_{4} = b_{3} + a_{4}$.

\textbf{STATEMENT-1 :} The numbers $b_{1},b_{2},b_{3},b_{4}$ are neither in A.P. nor in G.P. and \textbf{STATEMENT-2 :} The numbers $b_{1},b_{2},b_{3},b{4}$ are in H.P.\hfill[2008]

\begin{enumerate}
    

\item STATEMENT-1 is True, STATEMENT-2 IS True; STATEMENT-2 is the correct explanation for STATEMENT-1

\item STATEMENT-1 is True, STATEMENT-2 is True;

STATEMENT-2 is \textbf{NOT} a correct explanation for STATEMENT-1

\item STATEMENT-1 is True, STATEMENT-2 is False

\item STATEMENT-1 is False, STATEMENT-2 is True
\end{enumerate}
\end{enumerate}
\section{\textbf{Integer Value Correct Type}}

\begin{enumerate}
    

	\item Let $S_{k}, k = 1,2, \dots , 100$, denote the sum of the infinite geometric series whose first term is  $\frac{k - 1}{k!}$ and the common ratio is $\frac{1}{k}$. Then the value of $\frac{100^{2}}{100!}$ + $\sum\limits_{k=1}^{100} \abs{\brak{k^{2} - 3k +1}S_{k}} $ is \hfill[2010]

\item let $a_{1},a_{2},a_{3},\dots , a_{11}$ be real numbers satisfying $a_{1}= 15, 27 - 2a_{2} > 0$ and $a_{k}=2a_{k-1} - a_{k-2}$ for $k=3,4 \dots 11$. If $\frac{a_{1}^{2} + a_{2}^{2} + \dots +a_{11}^{2}}{11} = 90$ then the value of $\frac{a_{1} + a_{2} +\dots +a_{11}}{11}$ is equal to \hfill(2010)

\item $a_{1}, a_{2}, a_{3}\dots a_{100}$ be an arithmetic progression with $a_{1}= 3$ and $S_{p} =\sum\limits_{i=1}^{p} a_{i},1\leq p\leq 100$. For an integer n with $1 \leq n \leq 20,4$let $m= 5n$. If $\frac{S_{m}}{S_{n}}$ does not depend on n, then $a_{2}$ is \hfill(2011)

   \item A pack contains $n$ cards numbered for 1 to $n$, two consecutive numbered cards are removed from the pack and then the sum of the on the remaining cards is 1224. If the smaller of the numbers on the removed cards is $k$, then $k - 20=$ \hfill(JEE.Adv.2013)

   \item Let $a,b,c$ be positive integers such that $\frac{b}{a}$ is an integer. If $a$,$b,c$ are in geometric progression and the arithmetic mean of $a,b,c$ is $b + 2$, then the value of $\frac{a^{2} + a - 14}{a + 1}$ is \hfill(JEE.Adv.2014)

   \item Suppose all the numbers of an arithmetic progression(A.P.) are natural numbers. If the ratio of the sum of the first seven terms to the sum of the first eleven terms is 6 : 11 and the seventh term lies between 130 and 140 then the common difference of the A.P. is 
   
    \hfill(JEE.Adv.2015)

   \item The coefficient of $x^{9}$ in the expansion of $\brak{1+x}\brak{1+x^{2}}\brak{1+x^({3}} \dots \brak{1+x^{100}} $ is 
   
   \hfill(JEE.Adv.2015)

   \item The sides of a right angled triangle are in arithmetic progression. If the triangle has area 24, what is the length of its smallest side? 
   
   \hfill(JEE.Adv.2018)

   \item Let $X$ be the set consisting of the first 2018 terms of the arithmetic progression 1, 6, 11,\dots , and $Y$ be the set consisting of the first 2018 terms of the arithmetic progression 9, 16,23\dots . Then, the number of elements in the set $ X \cup Y $ is \hfill(JEE.Adv.2018)
   \item Let AP$\brak{a;d}$ denote the set of all the terms of an infinite arithmetic progression with the first term $a$ and the common differnce $d>0$. If AP$\brak{1;3}$AP$\brak{2;5}$AP$\brak{3;7}+$AP$\brak{a;d}$ then $a + d$ equals \hfill(JEE.Adv.2019)
   
    \end{enumerate}
   \end{document}
